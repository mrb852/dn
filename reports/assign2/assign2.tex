%
% diku datanet ACM
%
\documentclass[a4paper,danish]{dnacm} % V0.1

\acmVolume{}
\acmNumber{}
\acmYear{2015}
\acmMonth{}
\acmArticleNum{}
\acmdoi{1}

\begin{document}

\markboth{Gang of Three}{Assignment 2, HTTP SERVER}

\title{Assignment 3 \\ P2P \\ Datanet 2015} % title

\author{Christian Enevoldsen
\affil{Department Of Science, Copenhagen University}
}

\terms{Programming, Experimentation}

\keywords{HTTP, SERVER, P2P, PEER, TRACKER}

\maketitle

\begin{abstract}

This document covers the design of a simple Peer program for the Kascade p2p protocol. It starts with a brief introduction to p2p followed by the design face. Next it discusses some of the bottlenecks of the program and finally answers some theoretical questions.

\end{abstract}

\section{Introduction}
Peer-to-peer (short, p2p) is a networking protocol for partitioning tasks among multiple peers. This is usually used for sharing data, without the need to have a central server. Data can be everything from files to game stats. In this experiment we are going to implement a simple peer that will be able to download and distribute a set of files.
 
\section{Design}

The peer is divided into 4 submodules: The Client, Kascade, Peers and Server.

\subsection {Client}
The client's responsibility is to keep track of Kascade files, update them and download the contents from other Peers.

\subsection{Kascade}
Kascade is mainly an abstraction of the cascade files. There's a kascade loader which helps loading in the kascade files and creates Kascade objects from them. Kascade objects contains all information stored in the file and also additional information such as peers.

\subsection{Peer}
The peer module is also an abstraction. A Peer object contains information about the peers that you get from the tracker.

\subsection{Server}
The server module instantiates an Express node server which will serve other Peers, such that they can download data from the client.

\subsection{Libaries and Frameworks}
The Peer uses 3rd party libraries and a few build in ones.

\begin{itemize}
\item express - for networking\\
\item fs - for reading and writing files\\
\item crypto - for hashing\\
\end{itemize}


\section{Bottlenecks}

The peer has troubles loading the kascade files sometimes. This might be due to the files not being closed or opened correctly. The workaround is to restart the computer or wait some minutes and try again. Files cannot be written or read by multiple threads so this has an impact on the download time. The Kascade file system is not very well compressed so this has a significant impact in the downloading time, since it takes some seconds to load a big file. 

\section{Theory Questions}

Is TCP the best choice for the tracker communication?\\\\

UDP is better because it's stateless and doesn't cause overhead on packets. Using UDP can reduce this overhead by approximately 50%

Is TCP the best choice for the peer communication?\\\\

When downloading data any package loss can cause significant errors, so since TCP is resending dropped packages this might be a better solution.

Is HTTP the best choice for the tracker communication?\\\\

For the purpose of this assignment HTTP works great, but in the real world it is too slow.

Is HTTP the best choice for the peer communication?\\\\

Direct socket stream with custom events would be better.




\end{document}
% End of dnacm.tex (April 2015)